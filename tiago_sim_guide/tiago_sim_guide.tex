\documentclass[12pt, a4paper]{article}
\usepackage[utf8]{inputenc}
 
\usepackage[danish]{babel}
\usepackage{blindtext}
\usepackage[hidelinks]{hyperref}
\usepackage{graphicx}
\usepackage{caption}
\usepackage{subcaption}
\usepackage{xcolor}
\usepackage{url}
%\usepackage[margin=1in]{geometry}
\usepackage{natbib} %[square,comma,numbers]{natbib}
\usepackage{float}
\usepackage{todonotes}

\usepackage{listings}
\lstset{basicstyle=\ttfamily,
  showstringspaces=false,
  commentstyle=\color{red},
  keywordstyle=\color{blue}
}

\usepackage{minted}

\newlength\tindent
\setlength{\tindent}{\parindent}
\setlength{\parindent}{0pt}
\renewcommand{\indent}{\hspace*{\tindent}}



\title{\textbf{Tiago simulation guide}}
\author{Jacob Nørgaard}
 
\begin{document}

\pagenumbering{Alph}
\begin{titlepage}
\clearpage\maketitle
\thispagestyle{empty}
\end{titlepage}
\pagenumbering{arabic}

\section{Getting started}

Use following link to install fresh Ubuntu and ROS:

\url{http://wiki.ros.org/Robots/TIAGo/Tutorials/Installation/InstallUbuntuAndROS}

And the following link to install Tiago simulatior:

\url{http://wiki.ros.org/Robots/TIAGo/Tutorials/Installation/TiagoSimulation}

You should now have a workspace at 

\begin{minted}[breaklines]{bash}
~/tiago_public_ws
\end{minted}

When using the public simulator you have two robot options when launching:
    
\begin{itemize}
	\item TIAGo Steel: in this configuration the end-effector is a parallel gripper
    \item TIAGo Titanium: the wrist has a 6-axis force/torque sensor and the end-effector is the under-actuated 5-finger Hey5 hand.
\end{itemize}
    
To lauch the simulator, run the following command in your workspace:
\begin{minted}[breaklines]{bash}
roslaunch tiago_gazebo tiago_gazebo.launch public_sim:=true robot:=steel
\end{minted}

or
\begin{minted}[breaklines]{bash}
roslaunch tiago_gazebo tiago_gazebo.launch public_sim:=true robot:=titanium
\end{minted} 

You can also launch the simulation in a small office environment with people in it:

\begin{minted}[breaklines]{bash}
roslaunch tiago_gazebo tiago_gazebo.launch public_sim:=true robot:=titanium world:=simple_office_with_people
\end{minted}

\section{Controlling the simulation}

\subsection{Controlling the robot with your keyboard}

Open two terminals and direct to your workspace in both and source the workspace:

\begin{minted}[breaklines]{bash}
cd ~/tiago_public_ws
source ./devel/setup.bash
\end{minted}

Launch the simulation:

\begin{minted}[breaklines]{bash}
roslaunch tiago_gazebo tiago_gazebo.launch public_sim:=true robot:=titanium world:=simple_office_with_people
\end{minted}

In the second terminal, run the following command:

\begin{minted}[breaklines]{bash}
rosrun key_teleop key_teleop.py
\end{minted}

In this terminal windows you can now push your arrowkeys and see the robot move in the simulation.

\subsection{Move through velocity commands}

The velocity commands are sent to the simulation through the topic mobile\_base\_controllercmd\_vel of type \textit{geometry\_msgs/Twist} which is composed of:

\begin{itemize}
\item \textit{geometry\_msgs/Vector3 linear}
\item \textit{geometry\_msgs/Vector3 angular}
\end{itemize}

and is specified by a linear and an angular velocity.

\textbf{How to:}

Once again direct to your workspace and source the workspace:

\begin{minted}[breaklines]{bash}
cd ~/tiago_public_ws
source ./devel/setup.bash
\end{minted}

And Launch the simulation:

\begin{minted}[breaklines]{bash}
roslaunch tiago_gazebo tiago_gazebo.launch public_sim:=true robot:=steel
\end{minted}

In the second terminal windows you can now send commands specifying the linear and angular velocity by either:

\begin{minted}[breaklines]{bash}
rostopic pub /mobile_base_controller/cmd_vel geometry_msgs/Twist "linear:
	x: 0.5
	y: 0.0
	z: 0.0
angular:
	x: 0.0
	y: 0.0
	z: 0.0" -r 3
\end{minted}

or

\begin{minted}[breaklines]{bash}
rostopic pub /mobile_base_controller/cmd_vel geometry_msgs/Twist -r 3 -- '[0.5,0.0,0.0]' '[0.0, 0.0, 0.0]'
\end{minted}

To make the robot move with a constant linear velocity of 0.5 m/s and -r denotes that the message is published 3 times per second.

The same can be done with angular velocity:

\begin{minted}[breaklines]{bash}
rostopic pub /mobile_base_controller/cmd_vel geometry_msgs/Twist "linear:
	x: 0.0
	y: 0.0
	z: 0.0
angular:
	x: 0.0
	y: 0.0
	z: 0.3" -r 3
\end{minted}

or 

\begin{minted}[breaklines]{bash}
rostopic pub /mobile_base_controller/cmd_vel geometry_msgs/Twist -r 3 -- '[0.0,0.0,0.0]' '[0.0, 0.0, 0.3]'
\end{minted}

to make the robot rotate around its own axis at 0.3 rad/s.

\bibliographystyle{ieeetr}%
% \bibliography{bib}

\end{document}

\iffalse
\begin{minted}[breaklines]{bash}
\end{minted}
\fi